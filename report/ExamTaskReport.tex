%%%%%%%%%%%%%%%%%%%%%%%%%%%%%%%%%%%%%%%%%
% Arsclassica Article
% LaTeX Template
% Version 1.1 (10/6/14)
%
% This template has been downloaded from:
% http://www.LaTeXTemplates.com
%
% Original author:
% Lorenzo Pantieri (http://www.lorenzopantieri.net) with extensive modifications by:
% Vel (vel@latextemplates.com)
%
% License:
% CC BY-NC-SA 3.0 (http://creativecommons.org/licenses/by-nc-sa/3.0/)
%
%%%%%%%%%%%%%%%%%%%%%%%%%%%%%%%%%%%%%%%%%

%----------------------------------------------------------------------------------------
%	PACKAGES AND OTHER DOCUMENT CONFIGURATIONS
%----------------------------------------------------------------------------------------

\documentclass[
10pt, % Main document font size
a4paper, % Paper type, use 'letterpaper' for US Letter paper
oneside, % One page layout (no page indentation)
%twoside, % Two page layout (page indentation for binding and different headers)
headinclude,footinclude, % Extra spacing for the header and footer
BCOR5mm, % Binding correction
]{scrartcl}

\input{structure.tex} % Include the structure.tex file which specified the document structure and layout

\usepackage{amsmath}
\usepackage{wrapfig}
\usepackage{float}
\usepackage{tabularx}

\hyphenation{Fortran hy-phen-ation} % Specify custom hyphenation points in words with dashes where you would like hyphenation to occur, or alternatively, don't put any dashes in a word to stop hyphenation altogether

%----------------------------------------------------------------------------------------
%	TITLE AND AUTHOR(S)
%----------------------------------------------------------------------------------------

\title{\normalfont\spacedallcaps{Simulation \& analysis of a university network setup}} % The article title

\author{\spacedlowsmallcaps{Alexander Hillmer \& Lotte Steenbrink}} % The article author(s) - author affiliations need to be specified in the AUTHOR AFFILIATIONS block

\date{} % An optional date to appear under the author(s)

%----------------------------------------------------------------------------------------

\begin{document}

%----------------------------------------------------------------------------------------
%	HEADERS
%----------------------------------------------------------------------------------------

\renewcommand{\sectionmark}[1]{\markright{\spacedlowsmallcaps{#1}}} % The header for all pages (oneside) or for even pages (twoside)
%\renewcommand{\subsectionmark}[1]{\markright{\thesubsection~#1}} % Uncomment when using the twoside option - this modifies the header on odd pages
\lehead{\mbox{\llap{\small\thepage\kern1em\color{halfgray} \vline}\color{halfgray}\hspace{0.5em}\rightmark\hfil}} % The header style

\pagestyle{scrheadings} % Enable the headers specified in this block

%----------------------------------------------------------------------------------------
%	TABLE OF CONTENTS & LISTS OF FIGURES AND TABLES
%----------------------------------------------------------------------------------------

\maketitle % Print the title/author/date block

\setcounter{tocdepth}{2} % Set the depth of the table of contents to show sections and subsections only

\tableofcontents % Print the table of contents


%----------------------------------------------------------------------------------------
%	ABSTRACT
%----------------------------------------------------------------------------------------

\section*{Abstract} % This section will not appear in the table of contents due to the star (\section*)

TODO!

\lipsum[1] % Dummy text

\newpage % Start the article content on the second page, remove this if you have a longer abstract that goes onto the second page

%----------------------------------------------------------------------------------------
%	INTRODUCTION
%----------------------------------------------------------------------------------------

\section{Network Setup and Implementation}

\begin{figure}[!ht]
  \centering
  \includegraphics[width=\textwidth]{Figures/trace_plot.png}
  \caption{The Network Setup as simulated. TODO: add actual image!!} \label{fig:network}
\end{figure}

To analyze the desired network setup, simulations were conducted. The simulated network was set up as illustrated in fig. \ref{fig:network}. The following section discusses the implementation of the simulation and its configurations.

\subsection{Applications}
The network in question contains four different ways of using the network: video conference, FTP upload, web browsing and CCTV. Since all of these traffic types have different characteristics, dedicated applications which model their behavior were created.

\begin{description}
\item[Video conference] The video conference traffic occurring bidirectionally between the professor's laptop and the video conference laptop was modeled as a UDP application periodically sending messages of a fixed size. Packets arriving with a delay greater than a fixed maximum delay were discarded and considered lost.
\item[FTP upload] The FTP upload traffic flowing from one student's laptop to the Internet was modeled as a TCP based application which sends as much data at a time as it is allowed by TCP's congestion control algorithms.
The size of the file uploaded using FTP was assumed to be endless, that is: the file upload continued over the course of the entire simulation, no matter the simulation time.
\item[Web browsing] The web browsing traffic generated by all web browsing laptops was modeled as a TCP application which sends HTTP requests of a fixed length to the Internet and receives varying-length responses. The idle period between two requests varies as well.
\item[CCTV] The application modeling the traffic flowing from the CCTV camera to the CCTV monitoring station is based on UDP. Similar to the video conference application, it also periodically sends messages of a fixed size and ignores messages that have been delayed for too long. Its message size is significantly bigger, though, as can be seen in table \ref{table:app_config}.
\end{description}

For a detailed list of specific configuration parameters, see section \ref{sec:conf_details}.

\subsection{Configuration Details}
\label{sec:conf_details}

\begin{table}[H]
\begin{tabularx}{\textwidth}{ l l }
\multicolumn{2}{ l }{\textbf{Video conference application}} \\
\hline
  Message length       & 1388 B payload + 12 B minimal RTP header \\
                       &  = 1400 B\\
  Packet send interval & 40 ms \\
  Maximum packet delay & 100 ms \\
\multicolumn{2}{ l }{\textbf{Web browsing application}} \\
\hline
  HTTP request length  & 8 KiB \\
  HTTP response length & exponential($\mu = 671539$) [see section \ref{subsec:traffic_mod}] \\
  Idle interval        & exponential(20s) \\
\multicolumn{2}{ l }{\textbf{CCTV application}} \\
\hline
  Message length       & 10KiB \\
  Packet send interval & 40 ms \\
  Maximum packet delay & 100 ms \\
\end{tabularx}
\caption{Application configuration parameters}
\label{table:app_config}
\end{table}

\begin{table}[H]
\begin{tabularx}{\textwidth}{ l l }
\multicolumn{2}{ l }{\textbf{Network}} \\
\hline
  PPP queue size                & 50 frames\\
  WLAN version \& bandwidth     & IEEE 802.11g, 54 Mbit/s \\
  Ethernet version \& bandwidth & IEEE 802.3u, 100 Mbit/s \\
  VDSL bandwidth                & 100 Mbit/s \\
  VDSL delay                    & Internet: 30 ms \\
                                & CCTV Monitoring: 0 ms \\
  DFN bandwidth \& delay        & 100 Mbit/s, 5 ms \\
  Maximum packet loss rate      & 5 \% \\
  (video conference \& CCTV)    & \\
\end{tabularx}
\caption{Network configuration parameters}
\label{table:app_config}
\end{table}

\begin{table}[H]
\begin{tabularx}{\textwidth}{ l l }
\multicolumn{2}{ l }{\textbf{Simulation}} \\
\hline
  Duration          & 1000 s \\
  Repetitions       & 15 \\
  Number of laptops & 1, 5, 10, 15, 20, 30, 40, 50 or 60
\end{tabularx}
\caption{Simulation configuration parameters}
\label{table:app_config}
\end{table}

\subsection{HTTP Traffic modeling}
\label{subsec:traffic_mod}
%TODO: Chi square stuff

To model the size of the HTTP responses, the given trace file was analyzed in order to find a probability distribution that represents their data well.

\begin{figure}[!ht]
  \centering
  \includegraphics[width=0.7\textwidth]{Figures/trace_plot.png}
  \caption{Frequency of packet sizes found in the trace file} \label{fig:first_plot}
\end{figure}

After plotting the frequency of packet sizes found in the trace file (see Fig. \ref{fig:first_plot}), it was assumed that either a Poisson- or an exponential distribution could be a suitable theoretical distribution. The overall mean of the trace data is 671539.
Using Pearson's $\chi^2$ goodness of fit test, it was determined that, when using an exponential distribution with mean 671539, the null hypothesis can be rejected with a 95\% confidence level:\\

% TODO: zahlen (interval number, interval size) genauer begründen?
First, the sample data was divided into 100 intervals, where intervals with $\leq 5$ values were merged with their neighboring interval until the resulting interval contained more than 5 values, resulting in $k = 45$ intervals of observed data at the end.\\

Then, Pearson's $\chi^2$ test for goodness of fit $\chi^2 = \sum \frac{(Observed - Expected)^2}{Expected}$ was performed for both the Poisson and exponential distribution. In both cases, $\lambda = 671539$ was used as the function's input value.\\

\textbf{For the exponential distribution,} the resulting values were
\begin{align*}
\chi^2 &= 61.1437\\
\text{p-value} &= 0.0222
\end{align*}
leading to a rejection of the null hypothesis as $p < \alpha = 0.05$.\\
\textbf{For the Poisson distribution,} $\chi^2 = 9.3751 \cdot e^{20}$, suggesting that it is not a good fit.\\

This suggests that the exponential distribution is suitable to generate HTTP response sizes during the simulation.

%TODO: say why poisson doesn't fit?

\subsection{Network behavior}
In the following, the network behavior in terms of traffic and bandwidth utilization will be approximated roughly.
This helps set up the simulation network as described in section \ref{sec:sim_plan} and enables the estimation of the maximum number of web browsing laptops (also needed for the simulation).

\subsubsection{CCTV bit rate}
When sending packets of 10 kB of CCTV data every 40 ms, a bit rate of 2 megabits per second is generated:

\begin{align*}
\text{number of packets per second} &\cdot \text{packet size} = \text{bit rate}\\
\frac{1000ms}{40ms} &\cdot 10kB \cdot 8 = 2Mbps
\end{align*}

\subsubsection{Video conference bit rate}
The professor's bidirectional RTP over UDP video conference generates a bit rate of 0,56 megabits per second. In each direction, a bit rate of 0,28 Mbps is achieved:

\begin{align*}
\text{number of packets per second} &\cdot \text{packet size} = \text{bit rate}\\
\frac{1000ms}{40ms} &\cdot 1400B \cdot 8 = 0,28 Mbps
\end{align*}

Where the 1400B packet size are 1388 B payload + 12 B minimal RTP header.\\

In conclusion, a total $2\text{Mbps} + 2 \cdot 0,56 \text{Mbps} = 2,56 \text{Mbps}$ of the point-to-point radio link's bandwidth are used constantly.

\subsubsection{FTP upload bit rate}
Since the FTP file transfer is a bulk upload rather than a continuous data stream, its bandwidth usage is not constant. Using TCP's congestion avoidance algorithms, the TCP connection taking care of the transfer will attempt to use as much of the available bandwidth as possible (until collisions occur). Therefore, the FTP traffic is not part of the network's baseline traffic.

%----------------------------------------------------------------------------------------
%	METHODS
%----------------------------------------------------------------------------------------

\section{Simulation Plan}
\label{sec:sim_plan}

The overall goal of the simulation is to evaluate how the previously described applications perform and what the bottlenecks of the network are. First we identified potentially changing parameters of the network. These are the variables we can modify between different simulation runs. Afterwards we identified critical performance characteristics of the applications and performance metrics of the potential bottlenecks.

\subsection{Simulation Parameters}
The first parameter to look at is the amount of web browsing clients. We expect a decreasing performance with the amount of clients rising.
The second part is the CCTV. To determine the influence of the CCTV on the network performance, we either include or exclude the CCTV from the simulation.

To determine the maximum number of clients to simulate, we looked at the scenario itself. An amount of 60 clients is a reasonable amount of students to be present in a $400m^{2}$ area and surfing at the same time, if the lecture requires web research. To analyze the network thoroughly though, we need to make sure that the behavior with more clients does not change significantly. We found that it is not expected, that the behavior of the system changes significantly at a higher amount of clients, so we are able to extrapolate several network properties from the simulation results.
To be able to extrapolate properly, we need to determine the incrementation of the amount of clients as well. To receive a well formed graph, we should work with a low increment, when the network behavior changes significantly. At low numbers, the traffic share is higher, therefore we need a low increment. At higher numbers we can increase it. The resulting numbers of clients we simulated are:
\begin{align*}
N \in \{1,5,10,15,20,30,40,50,60\}
\end{align*}
To find a suitable duration of the simulation, we need to examine the behavior over time. We identified the browsing clients as the only application, whose behavior is depending on time. The mean time between HTTP requests is 20 seconds and it is exponentially distributed. The mean reply length is 671539 bytes and it is exponentially distributed as well. To get good results, we should include a high range of possible values in the simulation time. To calculate a reasonable expected highest value to take into account, we used the CDF of the exponential distribution:
\begin{align*}
y = 1-e^{-{\frac{1}{\mu}x}}
\end{align*}
If we want to include 0.95\% of the random values, we need to calculate the upper bound of the waiting time and the reply length, by applying the CDF:
\begin{align*}
{-ln(0.05)} * 671539 B \approx 2MB
\end{align*}
\begin{align*}
{-ln(0.05)} * 20s \approx 60s
\end{align*}
Our simulation showed, that the throughput for 60 clients goes down to 80000 bps. The expected time from simulation start to the end of the first request is therefore:
\begin{align*}
\frac{8\frac{b}{B} * 2MB}{80000 bps} + 60s = 260s
\end{align*}
To allow more of these requests, we set the maximum simulation time to 1000 seconds.

Throughout the simulation, we looked at the confidence intervals of several averaged values. We found that a repetition of 15 for each simulation combination produces a sufficient confidence level.

For the final simulation we end up with 9 different client and 2 CCTV settings, resulting in 270 simulation runs with a simulation duration of 1000 seconds.

\subsection{Performance Characteristics}
In the previous section we determined changing parameters between the simulation runs. The performance characteristics are the simulation measurements influenced by those parameters. The overall procedure is to record the measurements for every combination of parameters several times. Thus we have a set of 15 measurements for each combination. This allows us to calculate confidence intervals for each measurement. In the following we identify the performance characteristics of the implemented applications.

\subsubsection{Video Conference}
The video conference is based on UDP. Therefore packets can be dropped by the network permanently, reducing the quality of service. Additionally packets can be delayed. If a packet arrives too late, it can't contribute to the video stream anymore and is thus considered lost. The limit for the packet delay is 100 ms. If the packet loss rate is higher than 5\%, the quality of the video conference is considered to be too bad. Thus the crucial characteristic to look at is the average packet loss rate of the video conference in both directions.
Another characteristic is the average delay. Although you could expect it to be below 100 ms, when the average packet loss rate is sufficient, it is still an important value to rate the quality of service.
\begin{align*}
\frac{\text{lost packets} + \text{discarded packets}}{\text{sent packets}}
\end{align*}
TODO: formula
TODO: packet loss bursts? -> Packet loss over time
\subsubsection{Web Browsing}
The web browsing follows a simple request response model. The requests are carried out in a single session each. The important characteristic is how long the user has to wait until a response arrives. The response time is depending on the length of the response. For a greater length the user can cope with a longer response time. Therefore the crucial characteristic is the throughput the web client actually achieves.
It is possible to measure the average throughput of the web clients. But this value is misleading, since the browsers do not use the medium at all times. To address this issue we measured the fraction of the simulation time a session was active. The calculation of the final throughput is thus:
\begin{align*}
\sum_{i=1}^{N}\frac{r_{i}}{f_{i} * T}, \\
\text{where}~N &= \text{number of clients,} \\
r &= \text{received bits,} \\
f &= \text{fraction of the overall time,} \\
T &= \text{overall simulation time}
\end{align*}
To estimate the average response time, we can then further divide the average response length by the just acquired value.
Todays response times for common home networks are usually below 5 seconds. For a temporary back-up solution such as the network at hand slightly higher response times can be considered acceptable. Therefore you can say that it is okay to wait for a response for 10 seconds.

\subsubsection{FTP Upload}
The FTP Upload is modeled as a continous TCP stream to the internet. It runs for the entire simulation. The only important characteristic is the average throughput. The user is just interested in the time the upload takes. We measured the average throughput by the total amount of bytes uploaded to the server divided by the simulation time.
\begin{align*}
\frac{\text{uploadedBits}}{\text{simTime}}
\end{align*}

\subsubsection{CCTV}
The CCTV is used to monitor abnormal behavior at the remote campus. There are two characteristics, which can be considered. At first the delay of stream increases the reaction time of the staff. If the information arrives at the monitor one second late, it does not influence the reaction time of the staff in a significant way. Therefore it is neglectable.
As for the conference the average packet loss rate influences the quality of the video stream. If the packet loss rate is higher as 5\% the video stream is considered to not be recognizable anymore. Therefore the packet loss rate should not exceed 5\%.

TODO: Ist der Stream ab 5\% wirklich nicht mehr zu gebrauchen?
\subsection{Bottleneck Identification}
Besides looking at the performance characteristics of the applications, it is advisable to look at potential bottlenecks. We identified the wireless network at the remote campus and the radio link between the remote campus and the main campus as potential bottlenecks. The theoretical maximum bandwidth of 56 Mbps of the wireless network and a 12 Mbps bandwidth of the radio link are much higher than the bandwidths of the remaining links. Thus it is not necessary to evaluate those.

\subsubsection{Radio Link}
To evaluate the radio link two different characteristics are relevant. It is useful to determine how much bandwidth of the link is unused. Therefore we need to measure the average throughput. To receive even more information, we can look at the throughputs in each direction of the link. So we can identify the potential application, which is causing the traffic.
In case the link is indeed a bottleneck, it is necessary to collect data about the packet drop rate. When the link is overloaded, the packets will be held in a queue. When this queue is full, incoming packets are dropped. We can use this drop rate to determine if the link is a bottleneck.
Since our applications require a sufficiently short packet delay, it is also useful to record the average queueing delay, to evaluate the influence of the link on the streaming delays.
All of these measurements can be done at the remote router and main router to distinct between traffic directions.

\subsubsection{Wireless Network}
The practical throughput of a wireless network is much lower than the advertised theoretical maximum. This is due to collisions on the channel. So the wireless network can still be considered as a potential bottleneck, while its theoretical maximum bandwidth 56 Mbps is much higher than the 12Mbps of the radiolink. To evaluate the wireless network channel, the same statistics as of the radio link can be considered.
Additionally it is useful to look at the number of collisions to have a good metric to assess the issues caused by the channel.

\section{Simulation Evaluation}
TODO: answer (for with/without CCTV, I guess):\\
• What is the average data rate of the FTP upload?\\
• What are the average download rates of the HTTP clients?\\
• How do the other applications influence the video call?\\
• Is the direct radio link rate sufficient? Is it required to pause the CCTV service during the video conference?\\
• Where are the main bottlenecks of each variant?\\
• Which further changes would you suggest to improve the QoS for the video call?\\

\subsection{Network with CCTV}


\subsection{Network without CCTV}

\section{Recommended Changes To The Network}
TODO: are there any things we'd recommend to be changed based on our simulations?

- mehr APs *auf unterschiedlichen channeln!* (da das WLAN medium sonst überlastet ist und zum bottleneck wird) -> ein AP pro 20 clients

\section{Conclusion}
TODO

%----------------------------------------------------------------------------------------
%   GLOSSARY ETC
%----------------------------------------------------------------------------------------
\newpage

\section{Distribution of tasks}

TODO: Aufgabenverteilung aufschlüsseln

simulation parameters: paramaters changed between simulation runs
performance characteristics: measurements of the evaluation
response time: time between browser request and fully received response
CDF: Cumulative distribution function
\listoffigures % Print the list of figures

\listoftables % Print the list of tables


%----------------------------------------------------------------------------------------
%   BIBLIOGRAPHY (if any)
%----------------------------------------------------------------------------------------

\renewcommand{\refname}{\spacedlowsmallcaps{References}} % For modifying the bibliography heading

\bibliographystyle{unsrt}

\bibliography{sample.bib} % The file containing the bibliography

%----------------------------------------------------------------------------------------

\end{document}