%%%%%%%%%%%%%%%%%%%%%%%%%%%%%%%%%%%%%%%%%
% Arsclassica Article
% LaTeX Template
% Version 1.1 (10/6/14)
%
% This template has been downloaded from:
% http://www.LaTeXTemplates.com
%
% Original author:
% Lorenzo Pantieri (http://www.lorenzopantieri.net) with extensive modifications by:
% Vel (vel@latextemplates.com)
%
% License:
% CC BY-NC-SA 3.0 (http://creativecommons.org/licenses/by-nc-sa/3.0/)
%
%%%%%%%%%%%%%%%%%%%%%%%%%%%%%%%%%%%%%%%%%

%----------------------------------------------------------------------------------------
%	PACKAGES AND OTHER DOCUMENT CONFIGURATIONS
%----------------------------------------------------------------------------------------

\documentclass[
10pt, % Main document font size
a4paper, % Paper type, use 'letterpaper' for US Letter paper
oneside, % One page layout (no page indentation)
%twoside, % Two page layout (page indentation for binding and different headers)
headinclude,footinclude, % Extra spacing for the header and footer
BCOR5mm, % Binding correction
]{scrartcl}

\input{structure.tex} % Include the structure.tex file which specified the document structure and layout

\usepackage{amsmath}
\usepackage{wrapfig}

\hyphenation{Fortran hy-phen-ation} % Specify custom hyphenation points in words with dashes where you would like hyphenation to occur, or alternatively, don't put any dashes in a word to stop hyphenation altogether

%----------------------------------------------------------------------------------------
%	TITLE AND AUTHOR(S)
%----------------------------------------------------------------------------------------

\title{\normalfont\spacedallcaps{Simulation \& analysis of a university network setup}} % The article title

\author{\spacedlowsmallcaps{Alexander Hillmer \& Lotte Steenbrink}} % The article author(s) - author affiliations need to be specified in the AUTHOR AFFILIATIONS block

\date{} % An optional date to appear under the author(s)

%----------------------------------------------------------------------------------------

\begin{document}

%----------------------------------------------------------------------------------------
%	HEADERS
%----------------------------------------------------------------------------------------

\renewcommand{\sectionmark}[1]{\markright{\spacedlowsmallcaps{#1}}} % The header for all pages (oneside) or for even pages (twoside)
%\renewcommand{\subsectionmark}[1]{\markright{\thesubsection~#1}} % Uncomment when using the twoside option - this modifies the header on odd pages
\lehead{\mbox{\llap{\small\thepage\kern1em\color{halfgray} \vline}\color{halfgray}\hspace{0.5em}\rightmark\hfil}} % The header style

\pagestyle{scrheadings} % Enable the headers specified in this block

%----------------------------------------------------------------------------------------
%	TABLE OF CONTENTS & LISTS OF FIGURES AND TABLES
%----------------------------------------------------------------------------------------

\maketitle % Print the title/author/date block

\setcounter{tocdepth}{2} % Set the depth of the table of contents to show sections and subsections only

\tableofcontents % Print the table of contents


%----------------------------------------------------------------------------------------
%	ABSTRACT
%----------------------------------------------------------------------------------------

\section*{Abstract} % This section will not appear in the table of contents due to the star (\section*)

TODO!

\lipsum[1] % Dummy text

\newpage % Start the article content on the second page, remove this if you have a longer abstract that goes onto the second page

%----------------------------------------------------------------------------------------
%	INTRODUCTION
%----------------------------------------------------------------------------------------

\section{Prerequisites and Network Setup}

In order to conduct a meaningful simulation, a few configurations and calculations had to be made. In the following, these calculations, their results and the resulting network setup will be described.

\subsection{Traffic modeling}
%TODO: Chi square stuff

To model the size of the HTTP responses, the given trace file was analyzed in order to find a probability distribution that represents their data well.

\begin{wrapfigure}{r}{0.6\textwidth}
  \begin{center}
    \includegraphics[width=0.58\textwidth]{Figures/trace_plot.png}
  \end{center}
  \caption{Frequency of packet sizes found in the trace file}
  \vspace{-0pt}
  \label{fig:first_plot}
\end{wrapfigure}

After plotting the frequency of packet sizes found in the trace file (see Fig. \ref{fig:first_plot}), it was assumed that either a Poisson- or an exponential distribution could be a suitable theoretical distribution. The overall mean of the trace data is 671539.
Using Pearson's $\chi^2$ goodness of fit test, it was determined that, when using an exponential distribution with mean 671539, the null hypothesis can be rejected with a 95\% confidence level:

% TODO: zahlen (interval number, interval size) genauer begründen?
First, the sample data was divided into 100 intervals, where intervals with $\leq 15$ values were merged with their neighboring interval until the resulting interval contained at more than 15 values.

% TODO: add actual chi square value calc

This suggests that the exponential distribution is suitable to generate HTTP response sizes during the simulation.

%TODO: say why poisson doesn't fit?

\subsection{Network behavior}
In the following, the network behavior in terms of traffic and bandwidth utilization will be approximated roughly.
This helps set up the simulation network as described in section \ref{sec:sim_plan} and enables the estimation of the maximum number of web browsing laptops (also needed for the simulation).

\subsubsection{CCTV bit rate}
When sending packets of 10 kB of CCTV data every 40 ms, a bit rate of 2 megabits per second is generated:

\begin{align*}
\text{<number of packets per second >} &\cdot \text{<packet size>} = \text{<bit rate>}\\
\frac{1000ms}{40ms} &\cdot 10kB \cdot 8 = 2Mbps
\end{align*}

\subsubsection{Video conference bit rate}
The professor's bidirectional RTP over UDP video conference generates a bit rate of 0,56 megabits per second. In each direction, a bit rate of 0,28 Mbps is achieved:

\begin{align*}
\text{<number of packets per second >} &\cdot \text{<packet size>} = \text{<bit rate>}\\
\frac{1000ms}{40ms} &\cdot 1400B \cdot 8 = 0,28 Mbps
\end{align*}

Where the 1400B packet size are 1388 B payload + 12 B minimal RTP header.

Therefore, a total $2\text{Mbps} + 2 \cdot 0,56 \text{Mbps} = 2,56 \text{Mbps}$ of bandwidth are used constantly.

\subsubsection{FTP upload bit rate}
Since the FTP file transfer is a bulk upload rather than a continuous data stream, its bandwidth usage is not constant. Using TCP's congestion avoidance algorithms, the TCP connection taking care of the transfer will attempt to use as much of the available bandwidth as possible (until collisions occur). Therefore, the FTP traffic is not part of the network's baseline traffic.


%----------------------------------------------------------------------------------------
%	METHODS
%----------------------------------------------------------------------------------------

\section{Simulation Plan}
\label{sec:sim_plan}

TODO: einzelne Abschnitte \& Idee dahinter beschreiben

\subsection{Network with CCTV}
\subsection{Network without CCTV}

\section{Simulation Evaluation}
TODO: answer (for with/without CCTV, I guess):\\
• What is the average data rate of the FTP upload?\\
• What are the average download rates of the HTTP clients?\\
• How do the other applications influence the video call?\\
• Is the direct radio link rate sufficient? Is it required to pause the CCTV service during the video conference?\\
• Where are the main bottlenecks of each variant?\\
• Which further changes would you suggest to improve the QoS for the video call?\\

\subsection{Network with CCTV}
\subsection{Network without CCTV}

\section{Recommended Changes To The Network}
TODO: are there any things we'd recommend to be changed based on our simulations?

\section{Conclusion}
TODO

%----------------------------------------------------------------------------------------
%   GLOSSARY ETC
%----------------------------------------------------------------------------------------
\newpage

\section{Distribution of tasks}

TODO: Aufgabenverteilung aufschlüsseln


\listoffigures % Print the list of figures

\listoftables % Print the list of tables


%----------------------------------------------------------------------------------------
%   BIBLIOGRAPHY (if any)
%----------------------------------------------------------------------------------------

\renewcommand{\refname}{\spacedlowsmallcaps{References}} % For modifying the bibliography heading

\bibliographystyle{unsrt}

\bibliography{sample.bib} % The file containing the bibliography

%----------------------------------------------------------------------------------------

\end{document}