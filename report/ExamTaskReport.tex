%%%%%%%%%%%%%%%%%%%%%%%%%%%%%%%%%%%%%%%%%
% Arsclassica Article
% LaTeX Template
% Version 1.1 (10/6/14)
%
% This template has been downloaded from:
% http://www.LaTeXTemplates.com
%
% Original author:
% Lorenzo Pantieri (http://www.lorenzopantieri.net) with extensive modifications by:
% Vel (vel@latextemplates.com)
%
% License:
% CC BY-NC-SA 3.0 (http://creativecommons.org/licenses/by-nc-sa/3.0/)
%
%%%%%%%%%%%%%%%%%%%%%%%%%%%%%%%%%%%%%%%%%

%----------------------------------------------------------------------------------------
%	PACKAGES AND OTHER DOCUMENT CONFIGURATIONS
%----------------------------------------------------------------------------------------

\documentclass[
10pt, % Main document font size
a4paper, % Paper type, use 'letterpaper' for US Letter paper
oneside, % One page layout (no page indentation)
%twoside, % Two page layout (page indentation for binding and different headers)
headinclude,footinclude, % Extra spacing for the header and footer
BCOR5mm, % Binding correction
]{scrartcl}

%%%%%%%%%%%%%%%%%%%%%%%%%%%%%%%%%%%%%%%%%
% Arsclassica Article
% Structure Specification File
%
% This file has been downloaded from:
% http://www.LaTeXTemplates.com
%
% Original author:
% Lorenzo Pantieri (http://www.lorenzopantieri.net) with extensive modifications by:
% Vel (vel@latextemplates.com)
%
% License:
% CC BY-NC-SA 3.0 (http://creativecommons.org/licenses/by-nc-sa/3.0/)
%
%%%%%%%%%%%%%%%%%%%%%%%%%%%%%%%%%%%%%%%%%

%----------------------------------------------------------------------------------------
%	REQUIRED PACKAGES
%----------------------------------------------------------------------------------------

\usepackage[
nochapters, % Turn off chapters since this is an article        
beramono, % Use the Bera Mono font for monospaced text (\texttt)
eulermath,% Use the Euler font for mathematics
pdfspacing, % Makes use of pdftex’ letter spacing capabilities via the microtype package
dottedtoc % Dotted lines leading to the page numbers in the table of contents
]{classicthesis} % The layout is based on the Classic Thesis style

\usepackage{arsclassica} % Modifies the Classic Thesis package

\usepackage[T1]{fontenc} % Use 8-bit encoding that has 256 glyphs

\usepackage[utf8]{inputenc} % Required for including letters with accents

\usepackage{graphicx} % Required for including images
\graphicspath{{Figures/}} % Set the default folder for images

\usepackage{enumitem} % Required for manipulating the whitespace between and within lists

\usepackage{lipsum} % Used for inserting dummy 'Lorem ipsum' text into the template

\usepackage{subfig} % Required for creating figures with multiple parts (subfigures)

\usepackage{amsmath,amssymb,amsthm} % For including math equations, theorems, symbols, etc

\usepackage{varioref} % More descriptive referencing

%----------------------------------------------------------------------------------------
%	THEOREM STYLES
%---------------------------------------------------------------------------------------

\theoremstyle{definition} % Define theorem styles here based on the definition style (used for definitions and examples)
\newtheorem{definition}{Definition}

\theoremstyle{plain} % Define theorem styles here based on the plain style (used for theorems, lemmas, propositions)
\newtheorem{theorem}{Theorem}

\theoremstyle{remark} % Define theorem styles here based on the remark style (used for remarks and notes)

%----------------------------------------------------------------------------------------
%	HYPERLINKS
%---------------------------------------------------------------------------------------

\hypersetup{
%draft, % Uncomment to remove all links (useful for printing in black and white)
colorlinks=true, breaklinks=true, bookmarks=true,bookmarksnumbered,
urlcolor=webbrown, linkcolor=RoyalBlue, citecolor=webgreen, % Link colors
pdftitle={}, % PDF title
pdfauthor={\textcopyright}, % PDF Author
pdfsubject={}, % PDF Subject
pdfkeywords={}, % PDF Keywords
pdfcreator={pdfLaTeX}, % PDF Creator
pdfproducer={LaTeX with hyperref and ClassicThesis} % PDF producer
} % Include the structure.tex file which specified the document structure and layout

\usepackage{amsmath}

\hyphenation{Fortran hy-phen-ation} % Specify custom hyphenation points in words with dashes where you would like hyphenation to occur, or alternatively, don't put any dashes in a word to stop hyphenation altogether

%----------------------------------------------------------------------------------------
%	TITLE AND AUTHOR(S)
%----------------------------------------------------------------------------------------

\title{\normalfont\spacedallcaps{Simulation \& analysis of a university network setup}} % The article title

\author{\spacedlowsmallcaps{Alexander Hillmer \& Lotte Steenbrink}} % The article author(s) - author affiliations need to be specified in the AUTHOR AFFILIATIONS block

\date{} % An optional date to appear under the author(s)

%----------------------------------------------------------------------------------------

\begin{document}

%----------------------------------------------------------------------------------------
%	HEADERS
%----------------------------------------------------------------------------------------

\renewcommand{\sectionmark}[1]{\markright{\spacedlowsmallcaps{#1}}} % The header for all pages (oneside) or for even pages (twoside)
%\renewcommand{\subsectionmark}[1]{\markright{\thesubsection~#1}} % Uncomment when using the twoside option - this modifies the header on odd pages
\lehead{\mbox{\llap{\small\thepage\kern1em\color{halfgray} \vline}\color{halfgray}\hspace{0.5em}\rightmark\hfil}} % The header style

\pagestyle{scrheadings} % Enable the headers specified in this block

%----------------------------------------------------------------------------------------
%	TABLE OF CONTENTS & LISTS OF FIGURES AND TABLES
%----------------------------------------------------------------------------------------

\maketitle % Print the title/author/date block

\setcounter{tocdepth}{2} % Set the depth of the table of contents to show sections and subsections only

\tableofcontents % Print the table of contents


%----------------------------------------------------------------------------------------
%	ABSTRACT
%----------------------------------------------------------------------------------------

\section*{Abstract} % This section will not appear in the table of contents due to the star (\section*)

TODO!

\lipsum[1] % Dummy text

\newpage % Start the article content on the second page, remove this if you have a longer abstract that goes onto the second page

%----------------------------------------------------------------------------------------
%	INTRODUCTION
%----------------------------------------------------------------------------------------

\section{Prerequisites and Network Setup}

In order to conduct a meaningful simulation, a few configurations and calculations had to be made. In the following, these calculations, their results and the resulting network setup will be described.

\subsection{Traffic modeling}
TODO: Chi square stuff

\subsection{Network behavior}
In the following, the network behavior in terms of traffic and bandwidth utilization will be approximated roughly.
This helps set up the simulation network as described in \ref{sec:sim_plan} and enables the estimation of the maximum number of web browsing laptops (also needed for the simulation).

\subsubsection{CCTV bit rate}
When sending packets of 10 kB of CCTV data every 40 ms, a bit rate of 2 megabits per second is generated:

\begin{align*}
\text{<number of packets per second >} &\cdot \text{<packet size>} = \text{<bit rate>}\\
\frac{1000ms}{40ms} &\cdot 10kB \cdot 8 = 2Mbps
\end{align*}

\subsubsection{Video conference bit rate}
The professor's bidirectional RTP over UDP video conference generates a bit rate of 0,56 megabits per second. In each direction, a bit rate of 0,28 Mbps is achieved:

\begin{align*}
\text{<number of packets per second >} &\cdot \text{<packet size>} = \text{<bit rate>}\\
\frac{1000ms}{40ms} &\cdot 1400B \cdot 8 = 0,28 Mbps
\end{align*}

Where the 1400B packet size are 1388 B payload + 12 B minimal RTP header.

\subsubsection{FTP upload bit rate}
TODO


Therefore, a total $2\text{Mbps} + 2 \cdot 0,56 \text{Mbps} + \text{TODO: FTP} = 2,56 \text{TODO: update value} \text{Mbps}$ of bandwidth are used constantly.


%----------------------------------------------------------------------------------------
%	METHODS
%----------------------------------------------------------------------------------------

\section{Simulation Plan}
\label{sec:sim_plan}

TODO: einzelne Abschnitte \& Idee dahinter beschreiben

\section{Simulation Results}
TODO

\section{Recommended Changes To The Network}
TODO: are there any thing we'd recommend to be changed based on our simulations?

\section{Conclusion}
TODO

%----------------------------------------------------------------------------------------
%   GLOSSARY ETC
%----------------------------------------------------------------------------------------
\newpage

\listoffigures % Print the list of figures

\listoftables % Print the list of tables

%----------------------------------------------------------------------------------------
%   BIBLIOGRAPHY (if any)
%----------------------------------------------------------------------------------------

\renewcommand{\refname}{\spacedlowsmallcaps{References}} % For modifying the bibliography heading

\bibliographystyle{unsrt}

\bibliography{sample.bib} % The file containing the bibliography

%----------------------------------------------------------------------------------------

\end{document}